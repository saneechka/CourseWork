\documentclass[a4paper,14pt]{extarticle}

% Add language selection
\newif\ifenglish
\newcommand{\switchlanguage}[1]{%
    \if#1e\englishtrue\else\englishfalse\fi
}

% Default to Russian
\englishfalse

% Language-dependent commands
\newcommand{\thetitle}{%
    \ifenglish
        Fourier Transform and Its Applications
    \else
        Преобразование Фурье и его приложения
    \fi
}

\newcommand{\theauthor}{%
    \ifenglish
        Author: [Your Name]
    \else
        Автор: [Ваше имя]
    \fi
}

% Use these commands in your document
\title{\thetitle}
\author{\theauthor}

\section{Заключение}
В данной работе были рассмотрены различные аспекты применения преобразования Фурье и его производных в различных областях науки и техники. Были изучены теоретические основы преобразования Фурье, его свойства и методы вычисления. Также были рассмотрены примеры применения преобразования Фурье в обработке сигналов, изображений, анализе вибраций, решении дифференциальных уравнений и квантовой механике.

Преобразование Фурье является мощным инструментом для анализа и обработки данных, позволяющим перейти от временной области к частотной и обратно. Оно широко используется в различных областях науки и техники, таких как радиотехника, телекоммуникации, обработка изображений, анализ вибраций и многие другие.

В ходе работы были рассмотрены различные методы вычисления преобразования Фурье, включая дискретное преобразование Фурье (ДПФ) и быстрое преобразование Фурье (БПФ). Были изучены основные свойства преобразования Фурье, такие как линейность, задержка во времени, изменение масштаба, спектр производной и интеграла, теорема о свёртке и произведении сигналов.

Также были рассмотрены инструменты системы компьютерной алгебры Maple для работы с преобразованием Фурье, включая команды для вычисления прямого и обратного преобразования Фурье, дискретного преобразования Фурье и быстрого преобразования Фурье.

В заключение можно сказать, что преобразование Фурье является незаменимым инструментом для анализа и обработки данных в различных областях науки и техники. Оно позволяет эффективно решать задачи, связанные с анализом частотного спектра сигналов, обработкой изображений, анализом вибраций, решением дифференциальных уравнений и многими другими задачами.