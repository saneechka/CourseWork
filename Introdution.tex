\nocite{*}

\newpage
\thispagestyle{plain}  % Add this line to ensure page number appears
\section*{Введение}
\addcontentsline{toc}{section}{Введение}  % Add this to include in TOC without numbering
 

Данная курсовая работа посвящена комплексному исследованию теории преобразования Фурье, включая его дискретную форму, связь с преобразованием Лапласа, а также практическим приложениям в различных областях. Основной целью работы являлось изучение математического аппарата преобразования Фурье и его практической реализации для решения прикладных задач.

В рамках исследования было проведено детальное изучение интегралов и рядов Фурье, дискретного преобразования Фурье (ДПФ), быстрого преобразования Фурье (БПФ), а также их связи с преобразованием Лапласа. Практическая часть работы включала разработку алгоритмов и программных реализаций для вычисления прямого и обратного преобразования Фурье, спектрального анализа сигналов, фильтрации и обработки данных, а также сжатия изображений.

Особое внимание было уделено практическим исследованиям применения преобразования Фурье в обработке сигналов, анализе изображений, решении дифференциальных уравнений и работе с системами компьютерной алгебры. Структура работы включала логическое разделение на теоретическую часть, где подробно рассматривался математический аппарат преобразования Фурье, и практическую часть, демонстрирующую применение полученных знаний для решения конкретных задач с использованием современных компьютерных технологий и систем программирования.

Результаты исследования позволили углубить понимание теоретических основ преобразования Фурье и продемонстрировать его эффективность в решении широкого спектра практических задач.




