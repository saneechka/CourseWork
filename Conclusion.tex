
\nocite{*}

\newpage
\sectioncentered*{Вывод}
\addcontentsline{toc}{section}{Вывод}{

В данной курсовой работе было проведено
комплексное исследование преобразования Фурье,
 его теоретических основ и практических применений.
  В результате исследования был детально изучен математический аппарат преобразования Фурье,
   включая интегральное и дискретное преобразование, быстрое преобразование Фурье (БПФ),
    а также его связь с преобразованием Лапласа. Особое внимание было уделено основным свойствам и теоремам,
     лежащим в основе данного математического инструмента. В ходе работы были успешно исследованы практические
      аспекты применения преобразования Фурье в различных областях, включая обработку сигналов и изображений,
       спектральный анализ, решение дифференциальных уравнений и сжатие данных. В рамках практической части 
       были реализованы и протестированы алгоритмы вычисления прямого и обратного преобразования Фурье, 
       разработаны методы сжатия изображений с использованием ДПФ и реализована фильтрация сигналов в частотной области. 
       Проведенные эксперименты в системе компьютерной алгебры Maple наглядно продемонстрировали возможности разложения функций
        в ряд Фурье, вычисления преобразования Фурье и эффективность методов анализа и обработки сигналов.
         Полученные результаты подтверждают широкие возможности применения преобразования Фурье в различных
          областях науки и техники. Разработанные в ходе исследования алгоритмы и методы могут быть 
          эффективно использованы для решения практических задач цифровой обработки данных, что подтверждает
           практическую значимость проведенной работы. Результаты исследования создают основу для дальнейшего
            развития и совершенствования методов цифровой обработки сигналов и изображений на базе преобразования Фурье.


}



