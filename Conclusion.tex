\nocite{*}

\newpage
\section*{Заключение | Conclusion}
\addcontentsline{toc}{section}{Заключение | Conclusion}

\subsection*{Русская версия}
В данной курсовой работе было проведено
комплексное исследование преобразования Фурье,
 его теоретических основ и практических применений.
  В результате исследования был детально изучен математический аппарат преобразования Фурье,
   включая интегральное и дискретное преобразование, быстрое преобразование Фурье (БПФ),
    а также его связь с преобразованием Лапласа. Особое внимание было уделено основным свойствам и теоремам,
     лежащим в основе данного математического инструмента. В ходе работы были успешно исследованы практические
      аспекты применения преобразования Фурье в различных областях, включая обработку сигналов и изображений,
       спектральный анализ, решение дифференциальных уравнений и сжатие данных. В рамках практической части 
       были реализованы и протестированы алгоритмы вычисления прямого и обратного преобразования Фурье, 
       разработаны методы сжатия изображений с использованием ДПФ и реализована фильтрация сигналов в частотной области. 
       Проведенные эксперименты в системе компьютерной алгебры Maple наглядно продемонстрировали возможности разложения функций
        в ряд Фурье, вычисления преобразования Фурье и эффективность методов анализа и обработки сигналов.
         Полученные результаты подтверждают широкие возможности применения преобразования Фурье в различных
          областях науки и техники. Разработанные в ходе исследования алгоритмы и методы могут быть 
          эффективно использованы для решения практических задач цифровой обработки данных, что подтверждает
           практическую значимость проведенной работы. Результаты исследования создают основу для дальнейшего
            развития и совершенствования методов цифровой обработки сигналов и изображений на базе преобразования Фурье.

\subsection*{English version}
In this coursework, we conducted a comprehensive study of the Fourier transform, its theoretical foundations, and practical applications. The research thoroughly examined the mathematical apparatus of the Fourier transform, including integral and discrete transforms, Fast Fourier Transform (FFT), and its relationship with the Laplace transform. Special attention was paid to the fundamental properties and theorems underlying this mathematical tool. The work successfully investigated practical aspects of applying the Fourier transform in various fields, including signal and image processing, spectral analysis, differential equations solving, and data compression. The practical part implemented and tested algorithms for computing direct and inverse Fourier transforms, developed image compression methods using DFT, and implemented signal filtering in the frequency domain. Experiments conducted in the Maple computer algebra system clearly demonstrated the capabilities of Fourier series expansion, Fourier transform computation, and the effectiveness of signal analysis and processing methods. The obtained results confirm the wide possibilities of applying the Fourier transform in various fields of science and technology. The algorithms and methods developed during the research can be effectively used to solve practical digital data processing problems, which confirms the practical significance of the work. The research results create a foundation for further development and improvement of digital signal and image processing methods based on the Fourier transform.



